\documentclass[]{tufte-handout}

% ams
\usepackage{amssymb,amsmath}

\usepackage{ifxetex,ifluatex}
\usepackage{fixltx2e} % provides \textsubscript
\ifnum 0\ifxetex 1\fi\ifluatex 1\fi=0 % if pdftex
  \usepackage[T1]{fontenc}
  \usepackage[utf8]{inputenc}
\else % if luatex or xelatex
  \makeatletter
  \@ifpackageloaded{fontspec}{}{\usepackage{fontspec}}
  \makeatother
  \defaultfontfeatures{Ligatures=TeX,Scale=MatchLowercase}
  \makeatletter
  \@ifpackageloaded{soul}{
     \renewcommand\allcapsspacing[1]{{\addfontfeature{LetterSpace=15}#1}}
     \renewcommand\smallcapsspacing[1]{{\addfontfeature{LetterSpace=10}#1}}
   }{}
  \makeatother

\fi

% graphix
\usepackage{graphicx}
\setkeys{Gin}{width=\linewidth,totalheight=\textheight,keepaspectratio}

% booktabs
\usepackage{booktabs}

% url
\usepackage{url}

% hyperref
\usepackage{hyperref}

% units.
\usepackage{units}


\setcounter{secnumdepth}{-1}

% citations

% pandoc syntax highlighting
\usepackage{color}
\usepackage{fancyvrb}
\newcommand{\VerbBar}{|}
\newcommand{\VERB}{\Verb[commandchars=\\\{\}]}
\DefineVerbatimEnvironment{Highlighting}{Verbatim}{commandchars=\\\{\}}
% Add ',fontsize=\small' for more characters per line
\newenvironment{Shaded}{}{}
\newcommand{\AlertTok}[1]{\textcolor[rgb]{1.00,0.00,0.00}{\textbf{#1}}}
\newcommand{\AnnotationTok}[1]{\textcolor[rgb]{0.38,0.63,0.69}{\textbf{\textit{#1}}}}
\newcommand{\AttributeTok}[1]{\textcolor[rgb]{0.49,0.56,0.16}{#1}}
\newcommand{\BaseNTok}[1]{\textcolor[rgb]{0.25,0.63,0.44}{#1}}
\newcommand{\BuiltInTok}[1]{#1}
\newcommand{\CharTok}[1]{\textcolor[rgb]{0.25,0.44,0.63}{#1}}
\newcommand{\CommentTok}[1]{\textcolor[rgb]{0.38,0.63,0.69}{\textit{#1}}}
\newcommand{\CommentVarTok}[1]{\textcolor[rgb]{0.38,0.63,0.69}{\textbf{\textit{#1}}}}
\newcommand{\ConstantTok}[1]{\textcolor[rgb]{0.53,0.00,0.00}{#1}}
\newcommand{\ControlFlowTok}[1]{\textcolor[rgb]{0.00,0.44,0.13}{\textbf{#1}}}
\newcommand{\DataTypeTok}[1]{\textcolor[rgb]{0.56,0.13,0.00}{#1}}
\newcommand{\DecValTok}[1]{\textcolor[rgb]{0.25,0.63,0.44}{#1}}
\newcommand{\DocumentationTok}[1]{\textcolor[rgb]{0.73,0.13,0.13}{\textit{#1}}}
\newcommand{\ErrorTok}[1]{\textcolor[rgb]{1.00,0.00,0.00}{\textbf{#1}}}
\newcommand{\ExtensionTok}[1]{#1}
\newcommand{\FloatTok}[1]{\textcolor[rgb]{0.25,0.63,0.44}{#1}}
\newcommand{\FunctionTok}[1]{\textcolor[rgb]{0.02,0.16,0.49}{#1}}
\newcommand{\ImportTok}[1]{#1}
\newcommand{\InformationTok}[1]{\textcolor[rgb]{0.38,0.63,0.69}{\textbf{\textit{#1}}}}
\newcommand{\KeywordTok}[1]{\textcolor[rgb]{0.00,0.44,0.13}{\textbf{#1}}}
\newcommand{\NormalTok}[1]{#1}
\newcommand{\OperatorTok}[1]{\textcolor[rgb]{0.40,0.40,0.40}{#1}}
\newcommand{\OtherTok}[1]{\textcolor[rgb]{0.00,0.44,0.13}{#1}}
\newcommand{\PreprocessorTok}[1]{\textcolor[rgb]{0.74,0.48,0.00}{#1}}
\newcommand{\RegionMarkerTok}[1]{#1}
\newcommand{\SpecialCharTok}[1]{\textcolor[rgb]{0.25,0.44,0.63}{#1}}
\newcommand{\SpecialStringTok}[1]{\textcolor[rgb]{0.73,0.40,0.53}{#1}}
\newcommand{\StringTok}[1]{\textcolor[rgb]{0.25,0.44,0.63}{#1}}
\newcommand{\VariableTok}[1]{\textcolor[rgb]{0.10,0.09,0.49}{#1}}
\newcommand{\VerbatimStringTok}[1]{\textcolor[rgb]{0.25,0.44,0.63}{#1}}
\newcommand{\WarningTok}[1]{\textcolor[rgb]{0.38,0.63,0.69}{\textbf{\textit{#1}}}}

% longtable
\usepackage{longtable,booktabs}

% multiplecol
\usepackage{multicol}

% strikeout
\usepackage[normalem]{ulem}

% morefloats
\usepackage{morefloats}


% tightlist macro required by pandoc >= 1.14
\providecommand{\tightlist}{%
  \setlength{\itemsep}{0pt}\setlength{\parskip}{0pt}}

% title / author / date
\title{BST 140.651 Problem Set 2}
\author{Kai Aragaki}
\date{9/15/2020}


\begin{document}

\maketitle




\hypertarget{problem-1}{%
\section{Problem 1}\label{problem-1}}

\begin{center}\rule{0.5\linewidth}{0.5pt}\end{center}

\(Var(X) = E[X^2]-E[X]^2\)

Where \(Var(X) = E[(X-\mu)^2]\), so

\(E[(X-\mu)^2] = E[X^2]-E[X]^2\)

\(E[X^2-2X\mu + \mu^2] = E[X^2]-E[X]^2\)

\(E[X^2] + E[-2X\mu] + E[\mu^2] = E[X^2]-E[X]^2\)

Since \(\mu\) is a constant, and constants can be pulled out of the
expected value function,

\(E[X^2] -2\mu E[X] + \mu^2 E[1] = E[X^2] -E[X]^2\)

Simplifying, and with the knowledge that \(E[X] = \mu\):

\(-2\mu^2 + \mu^2 = -\mu^2\)

From which we find

\(-2\mu^2 = -2\mu^2\)

Which we know to be true.

\hypertarget{problem-2}{%
\section{Problem 2}\label{problem-2}}

\begin{center}\rule{0.5\linewidth}{0.5pt}\end{center}

\hypertarget{a}{%
\subsection{a}\label{a}}

\begin{center}\rule{0.5\linewidth}{0.5pt}\end{center}

\(g\) is a valid density if all values are positive and if the integral
across all values is 1.

We know all values are positive because it is the sum of three densities
(which themselves must always be positive), each of which is multiplied
by a scalar that is also positive.

To show that the integral across all values is equal to 1,

\(\int\limits_{-\infty}^{\infty}g(x) = 1\)

Since we know

\(g(x) = \pi_1f_1(x) + \pi_2f_2(x) + \pi_3f_3(x)\)

then

\(\int\limits_{-\infty}^{\infty}g(x) = \int\limits_{-\infty}^{\infty}\Big(\pi_1f_1(x) + \pi_2f_2(x) + \pi_3f_3(x)\Big)\)

which is equivalent to

\(\int\limits_{-\infty}^{\infty}g(x) = \pi_1\int\limits_{-\infty}^{\infty}f_1(x) + \pi_2\int\limits_{-\infty}^{\infty}f_2(x)+ \pi_3\int\limits_{-\infty}^{\infty}f_3(x)\)

and since each function \(f_i\) is a valid density their integrals must
equal 1, giving

\(\int\limits_{-\infty}^{\infty}g(x) = \pi_1 + \pi_2 + \pi_3\)

Since we are additionally told \(\Sigma^3_{i=1}\pi_i = 1\)

\(\int\limits_{-\infty}^{\infty}g(x) = 1\)

\begin{marginfigure}
\(\blacksquare\)
\end{marginfigure}

\hypertarget{b}{%
\subsection{b}\label{b}}

\begin{center}\rule{0.5\linewidth}{0.5pt}\end{center}

We know that mean (or expected value) for a continous distribution is
given by

\(\int\limits_{-\infty}^{\infty}xg(x)\)

Which is

\(\int\limits_{-\infty}^{\infty}x\Big(\pi_1f_1(x)+\pi_2f_2(x)+\pi_3f_3(x)\Big)\)

which can be expanded to

\(\pi_1\int\limits_{-\infty}^{\infty}xf_1(x) + \pi_2\int\limits_{-\infty}^{\infty}xf_2(x)+\pi_3\int\limits_{-\infty}^{\infty}xf_3(x)\)

And since we know the definition of the expected value, we can rewrite
this as

\(E[X] = \pi_1\mu_1 + \pi_2\mu_2+\pi_3\mu_3\)

\hypertarget{c}{%
\subsection{c}\label{c}}

\begin{center}\rule{0.5\linewidth}{0.5pt}\end{center}

We have previously shown that the definition of

\(Var(x) = E[X^2] - E[X]^2\)

We can represent the variance of \(g\) in these terms:

\(\int\limits_{-\infty}^{\infty}x^2g(x) - [\int\limits_{-\infty}^{\infty}xg(x)]^2\)

The right term we have previously found - it's simply the square of the
expectation value we found in the previous problem. The left terms can
be found by first writing \(g\) in terms of \(f\)

\(\int\limits_{-\infty}^{\infty}x^2g(x) = \int\limits_{-\infty}^{\infty}x^2\Big(\pi_1f_1(x)+\pi_2f_2(x)+\pi_3f_3(x)\Big)\)

or distributed:

\(\int\limits_{-\infty}^{\infty}x^2g(x) = \pi_1\int\limits_{-\infty}^{\infty}x^2f_1(x) + \pi_2\int\limits_{-\infty}^{\infty}x^2f_2(x) + \pi_3\int\limits_{-\infty}^{\infty}x^2f_3(x)\)

and then noticing that the variance equation can be written as follows:

\(\sigma^2 = E[X^2] + \mu^2\)

And solving for \(E[X^2]\):

\(E[X^2] = \mu^2 + \sigma^2\)

which is equivalent to

\(\int\limits_{-\infty}^{\infty}x^2f_i(x) = \mu^2_i + \sigma^2_i\)

Substituting in, we get

\(\int\limits_{-\infty}^{\infty}x^2g(x) = \pi_1(\mu^2_1 + \sigma^2_1) +\pi_2(\mu^2_2 + \sigma^2_2) +\pi_3(\mu^2_3 + \sigma^2_3)\)

Combining this left hand term with the right hand term, we get

\(Var(x) = \pi_1(\mu^2_1 + \sigma^2_1) +\pi_2(\mu^2_2 + \sigma^2_2) +\pi_3(\mu^2_3 + \sigma^2_3) -(\pi_1\mu_1 + \pi_2\mu_2+\pi_3\mu_3)^2\)

\hypertarget{problem-3}{%
\section{Problem 3}\label{problem-3}}

\begin{center}\rule{0.5\linewidth}{0.5pt}\end{center}

\hypertarget{a-1}{%
\subsection{a}\label{a-1}}

\begin{center}\rule{0.5\linewidth}{0.5pt}\end{center}

The mean of a continuous distribution is

\(\int\limits_{lower}^{upper}xf(x)\)

In this case the lower and upper bounds are given as 0 and 1
respectively.

Therefore the mean is given by

\(\int\limits_{0}^{1}(k+1)x^{k+1}\)

\(\int\limits_{0}^{1}kx^{k+1} + \int\limits_{0}^{1}x^{k+1}\)

\(\frac{kx^{k+2}}{k+2}\Bigg|^1_0 + \frac{x^{k+2}}{k+2}\Bigg|^1_0\)

\(\frac{k}{k+2} + \frac{1}{k+2}\)

\(\frac{k+1}{k+2}\)

\hypertarget{b-1}{%
\subsection{b}\label{b-1}}

\begin{center}\rule{0.5\linewidth}{0.5pt}\end{center}

The variance of this distribution is

\(Var(x) = E[X^2] - E[X]^2\)

We know \(E[X]=\mu\), which is what we found in part \(a\), so squaring
it is trivial. Additionally,

\(E[X^2] = \int\limits_{0}^{1}x^2\big((k+1)x^k\big) = \int\limits_{0}^{1}(k+1)x^{k+2}\)

And by distributing,

\(\int\limits_{0}^{1}kx^{k+2} + \int\limits_{0}^{1}x^{k+2}\)

We can see by analogy to the first problem that this will integrate and
simplify to

\(\frac{k+1}{k+3}\)

\hypertarget{problem-4}{%
\section{Problem 4}\label{problem-4}}

\begin{center}\rule{0.5\linewidth}{0.5pt}\end{center}

\hypertarget{a-2}{%
\subsection{a}\label{a-2}}

\begin{center}\rule{0.5\linewidth}{0.5pt}\end{center}

To calculate the mean, we know that the mean is, for the limits of this
given function

\(E[X] = \int\limits^\infty_{0}xf(x) = \frac{1}{3.3}\int\limits^\infty_{0}xe^{\frac{-x}{3.3}}\)

We can integrate by parts using - that is, we know the following to be
true:

\(\int udv = uv - \int vdu\)

so if we set \(u = x\), \(du = dx\), \(v = -3.3e^{\frac{-x}{3.3}}\), and
\(dv = e^{\frac{-x}{3.3}}dx\),

\(\int xe^{\frac{-x}{3.3}}= 3.3xe^{\frac{-x}{3.3}} + 3.3\int e^{\frac{-x}{3.3}}dx = 3.3xe^{\frac{-x}{3.3}} - 3.3^2e^{\frac{-x}{3.3}}\)

Substituting this integral back into the original, we obtain

\(xe^{\frac{-x}{3.3}} - 3.3e^{\frac{-x}{3.3}}\)

evaluating at its limits of \(0\) and \(\infty\), and with the knowledge
that \(e^{-x}\) approaches 0 much faster than \(x\) approaches infinity,

\(xe^{\frac{-x}{3.3}} - 3.3e^{\frac{-x}{3.3}}\Big|^\infty_0 = [0 + 0] - [0 - 3.3] = 3.3\)

\begin{marginfigure}
\(\blacksquare\)
\end{marginfigure}

To calculate the variance, we solve the following:

\(Var(X) = \int\limits^\infty_{0}x^2f(x) - E[X]^2\)

But since we know the value of \(E[X]\) (and squaring it is trivial) we
focus our efforts on the left hand side of the equation:

\(\int\limits^\infty_{0}x^2f(x) = \frac{1}{3.3}\int\limits^\infty_{0}x^2e^{\frac{-x}{3.3}}\)

As before, we begin to integrate by parts, with \(u = x^2\),
\(du = 2xdx\), \(v = -3.3e^{\frac{-x}{3.3}}\), and
\(dv = e^{\frac{-x}{3.3}}\):

\(\frac{1}{3.3}[-3.3x^2e^{\frac{-x}{3.3}}+6.6\int e^{\frac{-x}{3.3}}xdx]\)

We integrate by parts again, this time where \(u = x\), \(du = dx\),
\(v = -3.3e^{\frac{-x}{3.3}}\), and \(dv = e^{\frac{-x}{3.3}}dx\),

\(\frac{1}{3.3}\Big[-3.3x^2e^{\frac{-x}{3.3}}+6.6[-3.3xe^{\frac{-x}{3.3}} + 3.3 \int e^{\frac{-x}{3.3}}dx]\Big]\)

Which is equivalent to

\(\frac{1}{3.3}\Big[-3.3x^2e^{\frac{-x}{3.3}}+6.6[-3.3xe^{\frac{-x}{3.3}} - 3.3^2 e^{\frac{-x}{3.3}}]\Big]\)

As before, evaluating at the limits \(0\) to \(\infty\) and substitute
back in \(E[X]^2\)

\(Var(X) = 2(3.3)^2-3.3^2 = 3.3^2 = 10.89\)

\begin{marginfigure}
\(\blacksquare\)
\end{marginfigure}

\hypertarget{b-2}{%
\subsection{b}\label{b-2}}

\begin{center}\rule{0.5\linewidth}{0.5pt}\end{center}

If we substitute \(3.3\) for \(\beta\) in our previous solutions, we
find

\(E[X] = xe^{\frac{-x}{\beta}} - \beta e^{\frac{-x}{\beta}}\Big|^\infty_0 = [0 + 0] - [0 - \beta] = \beta\)

and

\(Var(X) = \frac{1}{\beta}\Big[-\beta x^2e^{\frac{-x}{\beta}}+2\beta[-\beta xe^{\frac{-x}{\beta}} - \beta^2 e^{\frac{-x}{\beta}}]\Big] - \beta^2 = \beta^2\)

\hypertarget{c-1}{%
\subsection{c}\label{c-1}}

\begin{center}\rule{0.5\linewidth}{0.5pt}\end{center}

\begin{Shaded}
\begin{Highlighting}[]
\NormalTok{tib <-}\StringTok{ }\KeywordTok{expand_grid}\NormalTok{(}\DataTypeTok{x =} \KeywordTok{seq}\NormalTok{(}\DecValTok{0}\NormalTok{, }\DecValTok{10}\NormalTok{, }\DataTypeTok{by =} \FloatTok{0.01}\NormalTok{), }\DataTypeTok{b =} \KeywordTok{c}\NormalTok{(}\FloatTok{0.1}\NormalTok{, }\DecValTok{1}\NormalTok{, }\DecValTok{10}\NormalTok{)) }\OperatorTok\StringTok{ }\KeywordTok{mutate}\NormalTok{(}\DataTypeTok{y =}\NormalTok{ (}\DecValTok{1}\OperatorTok{/}\NormalTok{b) }\OperatorTok{*}\StringTok{ }
\StringTok{    }\KeywordTok{exp}\NormalTok{(}\OperatorTok{-}\NormalTok{x}\OperatorTok{/}\NormalTok{b))}

\KeywordTok{ggplot}\NormalTok{(tib, }\KeywordTok{aes}\NormalTok{(}\DataTypeTok{x =}\NormalTok{ x, }\DataTypeTok{y =}\NormalTok{ y, }\DataTypeTok{color =} \KeywordTok{as.factor}\NormalTok{(b))) }\OperatorTok{+}\StringTok{ }\KeywordTok{scale_color_viridis_d}\NormalTok{(}\DataTypeTok{end =} \FloatTok{0.8}\NormalTok{) }\OperatorTok{+}\StringTok{ }
\StringTok{    }\KeywordTok{geom_line}\NormalTok{()}
\end{Highlighting}
\end{Shaded}

\includegraphics{/Users/kaiaragaki/Documents/github-repos/Biostats_140.651.01/output/hw_2_files/figure-latex/unnamed-chunk-4-1}

\hypertarget{problem-5}{%
\section{Problem 5}\label{problem-5}}

\begin{center}\rule{0.5\linewidth}{0.5pt}\end{center}

\hypertarget{a-3}{%
\subsection{a}\label{a-3}}

\begin{center}\rule{0.5\linewidth}{0.5pt}\end{center}

We know the mean to be

\(\int\limits_0^\infty xf(x)dx = \int\limits_0^\infty \frac{1}{\beta^\alpha\Gamma(\alpha)}x^\alpha e^{\frac{-x}{\beta}}dx\)

We begin by substituting, where \(u = \frac{x}{\beta}\) and
\(du = \frac{dx}{\beta}\), which implies \(x = \beta u\) and
\(dx = \beta du\):

\(\int\limits_0^\infty \frac{1}{\beta^\alpha\Gamma(\alpha)}(\beta u)^\alpha e^{-u}\beta du = \int\limits_0^\infty \frac{1}{\Gamma(\alpha)}u^\alpha e^{-u}\beta du = \frac{\Gamma(\alpha+1)}{\Gamma(\alpha)}\beta\)

Since we know that \(\Gamma(\alpha) = (\alpha-1)!\), then

\(\frac{\Gamma(\alpha+1)}{\Gamma(\alpha)}\beta = \alpha\beta\)

\begin{marginfigure}
\(\blacksquare\)
\end{marginfigure}

For the variance, we know

\(Var(X) = E[X^2] - E[X]^2\)

We begin to solve for the first portion:

\(E[X^2] = \int \limits _0^\infty x^2f(x) = \int\limits_0^\infty \frac{1}{\beta^\alpha\Gamma(\alpha)}x^{\alpha+1} e^{\frac{-x}{\beta}}dx\)

Using the same substitution as above, we can write it as

\(\int\limits_0^\infty \frac{1}{\beta^\alpha\Gamma(\alpha)}(\beta u)^{\alpha+1} e^u \beta du = \int\limits_0^\infty \frac{1}{\Gamma(\alpha)}\beta^2u^{\alpha+1} e^udu = \frac{\Gamma(\alpha+2)}{\Gamma(\alpha)}\beta^2 = (\alpha + 1)\alpha\beta^2 = \alpha^2\beta^2 + \alpha\beta^2\)

Solving for second portion is as simple as squaring the value we found
for the mean, so in total,

\(Var(X) = E[X^2] - E[X]^2 = \alpha^2\beta^2 + \alpha\beta^2 - \alpha^2\beta^2 = \alpha\beta^2\)

\begin{marginfigure}
\(\blacksquare\)
\end{marginfigure}

\hypertarget{b-3}{%
\subsection{b}\label{b-3}}

\begin{center}\rule{0.5\linewidth}{0.5pt}\end{center}

We can simply substitute in these values for \(\alpha\) and \(\beta\):

\(\mu = \alpha\beta = \frac{2p}{2} = p\)

\(\sigma^2 = \alpha\beta^2 = \frac{4p}{2} = 2p\)

\hypertarget{problem-6}{%
\section{Problem 6}\label{problem-6}}

\begin{center}\rule{0.5\linewidth}{0.5pt}\end{center}

\hypertarget{a-4}{%
\subsection{a}\label{a-4}}

\begin{center}\rule{0.5\linewidth}{0.5pt}\end{center}

\(E[X] = \int \limits_0^1xf(x) = \int \limits_0^1 \frac{x^\alpha(1-x)^{\beta-1}}{B(\alpha, \beta)}\)

since we know

\(B(\alpha, \beta) = \int\limits^1_0 x^{\alpha-1}(1-x)^{\beta-1}dx = \frac{\Gamma(\alpha)\Gamma(\beta)}{\Gamma(\alpha + \beta)}\)

then

\(B(\alpha + 1, \beta) = \int\limits^1_0 x^{\alpha}(1-x)^{\beta-1}dx = \frac{\Gamma(\alpha + 1)\Gamma(\beta)}{\Gamma(\alpha + 1 + \beta)}\)

so

\(\int \limits_0^1 \frac{x^\alpha(1-x)^{\beta-1}}{B(\alpha, \beta)} = \frac{\Gamma(\alpha + 1)\Gamma(\beta)}{\Gamma(\alpha + 1 + \beta)} \times \frac{\Gamma(\alpha + \beta)}{\Gamma(\alpha)\Gamma(\beta)} = \frac{\alpha\Gamma(\alpha)\Gamma(\beta)}{(\alpha + \beta)\Gamma(\alpha + \beta)} \times \frac{\Gamma(\alpha + \beta)}{\Gamma(\alpha)\Gamma(\beta)} = \frac{\alpha}{\alpha + \beta}\)

\begin{marginfigure}
\(\blacksquare\)
\end{marginfigure}

\hypertarget{b-4}{%
\subsection{b}\label{b-4}}

\begin{center}\rule{0.5\linewidth}{0.5pt}\end{center}

For the variance, like above we solve in portions, starting with the
first portion:

\(E[X^2] = \int \limits_0^1x^2f(x) = \int \limits_0^1 \frac{x^{\alpha+1}(1-x)^{\beta-1}}{B(\alpha, \beta)} = \frac{\Gamma(\alpha + 2)\Gamma(\beta)}{\Gamma(\alpha + 2 + \beta)} \times \frac{\Gamma(\alpha + \beta)}{\Gamma(\alpha)\Gamma(\beta)} = \frac{(\alpha + 1)(\alpha)\Gamma(\alpha)\Gamma(\beta)}{(\alpha + \beta + 1)(\alpha+\beta)\Gamma(\alpha + \beta)} \times \frac{\Gamma(\alpha + \beta)}{\Gamma(\alpha)\Gamma(\beta)} = \frac{\alpha(\alpha+1)}{(\alpha + \beta + 1)(\alpha + \beta)}\)

The second portion is simply the mean squared, so the variance becomes

\(Var(X) = \frac{\alpha(\alpha+1)}{(\alpha + \beta + 1)(\alpha + \beta)} - \Big(\frac{\alpha}{\alpha + \beta}\Big)^2 = \frac{\alpha(\alpha+1)(\alpha + \beta) - (\alpha + \beta + 1)\alpha^2}{(\alpha + \beta + 1)(\alpha + \beta)^2} = \frac{\alpha\beta}{(\alpha + \beta + 1)(\alpha + \beta)^2}\)

\begin{marginfigure}
\(\blacksquare\)
\end{marginfigure}

\hypertarget{problem-7}{%
\section{Problem 7}\label{problem-7}}

\begin{center}\rule{0.5\linewidth}{0.5pt}\end{center}

\hypertarget{a-5}{%
\subsection{a}\label{a-5}}

\begin{center}\rule{0.5\linewidth}{0.5pt}\end{center}

The mean for discrete functions can be written as

\(E[X] = \sum\limits_{x = 0}^{\infty}xp(x) = \sum\limits_{x = 0}^{\infty}\frac{xe^{-\lambda}\lambda^x}{x!}\)

We know \(e^{-\lambda}\) is constant, so we can pull it out of the sum:

\(e^{-\lambda}\sum\limits_{x = 0}^{\infty}\frac{x\lambda^x}{x!}\)

We further note that the first term is 0 - so we can safely start the
sum at \(x = 1\):

\(e^{-\lambda}\sum\limits_{x = 1}^{\infty}\frac{x\lambda^x}{x!}\)

We now note two things: Each term in our summation has at least one
\(\lambda\), which can be factored out. Additionally,
\(\frac{x}{x!} = \frac{1}{(x-1)!}\), so the sum can be rewritten as

\(e^{-\lambda}\lambda\sum\limits_{x = 1}^{\infty}\frac{\lambda^{x-1}}{(x-1)!}\)

By performing a change of variable, where \(y = x - 1\), we can more
easily see a familiar infinite sum:

\(e^{-\lambda}\lambda\sum\limits_{y = 1}^{\infty}\frac{\lambda^{y}}{y!} = e^{-\lambda}e^\lambda\lambda = \lambda\)

\begin{marginfigure}
\(\blacksquare\)
\end{marginfigure}

\hypertarget{b-5}{%
\subsection{b}\label{b-5}}

\begin{center}\rule{0.5\linewidth}{0.5pt}\end{center}

For this problem, we are given a hint of \(E[X(X-1)]\). Let's evaluate
this:

\(E[X(X-1)] = \sum\limits_{x=0}^\infty \frac{x(x-1)e^{-\lambda}\lambda^x}{x!}\)

We initially note that the first two terms will be 0, so we can shift
our sum to start at 2:

\(\sum\limits_{x=2}^\infty \frac{x(x-1)e^{-\lambda}\lambda^x}{x!}\)

We also note again that \(e^{-\lambda}\) is constant and can be factored
out:

\(e^{-\lambda}\sum\limits_{x=2}^\infty \frac{x(x-1)\lambda^x}{x!}\)

Further still, we note that the first two terms of \(x!\) can be
cancelled out with the terms in the numerator:

\(e^{-\lambda}\sum\limits_{x=2}^\infty \frac{\lambda^x}{(x-2)!}\)

If we examine the first few terms, we will notice that each term has at
least \(\lambda^2\). We can factor this out as well:

\(e^{-\lambda}\lambda^2\sum\limits_{x=2}^\infty \frac{\lambda^{x-2}}{(x-2)!}\)

A change of variable (where \(y = x-2\)) lets us see that this is a
familiar sum:

\(e^{-\lambda}\lambda^2\sum\limits_{y=0}^\infty \frac{\lambda^{y}}{(y)!} = e^{-\lambda}e^\lambda\lambda^2 = \lambda^2\)

We know this to be \(E[X(X-1)]\). Manipulating this expression, we see

\(E[X(X-1)] = E[X^2-X]\)

Since we know expectation values can be separated, this is equivalent to

\(E[X^2] - E[X]\)

and we know \(E[X] = \lambda\) So we can solve for \(E[X^2]\):

\(\lambda^2 = E[X^2] - \lambda \implies E[X^2] = \lambda^2 + \lambda\)

So,

\(Var(X) =\lambda^2 + \lambda - E[X]^2 = \lambda^2 + \lambda - \lambda^2 = \lambda\)

\begin{marginfigure}
\(\blacksquare\)
\end{marginfigure}

\hypertarget{problem-8}{%
\section{Problem 8}\label{problem-8}}

\begin{center}\rule{0.5\linewidth}{0.5pt}\end{center}

\hypertarget{a-6}{%
\subsection{a}\label{a-6}}

\begin{center}\rule{0.5\linewidth}{0.5pt}\end{center}

We can write the uniform distribution as

\(f(x) = 1\)

So the expected value (for \(0 < x < 1\)) can be calculated as

\(E[X] = \int\limits^1_0 xdx = \frac{x^2}{2}\Big|^1_0 = \frac{1}{2}\)

Converting to a percentage, the expected percentage of a randomly
selected person's body that is covered in freckles is \(50\%\)

\hypertarget{b-6}{%
\subsection{b}\label{b-6}}

\begin{center}\rule{0.5\linewidth}{0.5pt}\end{center}

The variance is given by

\(Var(X) = E[X^2] - E[X]^2\)

Beginning with the first portion,

\(E[X^2] = \int\limits^1_0 x^2dx = \frac{x^3}{3}\Big|^1_0 = \frac{1}{3}\)

So,

\(Var(X) = \frac{1}{3} - \frac{1}{4} = \frac{1}{12}\)

\hypertarget{problem-9}{%
\section{Problem 9}\label{problem-9}}

\begin{center}\rule{0.5\linewidth}{0.5pt}\end{center}

\hypertarget{a-7}{%
\subsection{a}\label{a-7}}

\begin{center}\rule{0.5\linewidth}{0.5pt}\end{center}

If we consider \(X\) to be the number of songs listened to AFTER the
repeated song was listened to, then we know that \(P(X = 1) = 0\) -
listening to just one song cannot be a repeat of anything.

If we manage to listen to all the songs without a repeat, the next song
we listen to MUST be a repeat: there is no song in the list that exists
that we have not yet heard \(P(X = 1001) = 1\). Any additional songs are
not considered. So, \(2 \le X \le 1001\).

Let's think about what the probability of listening to 3 songs (and
hearing a repeat on the third song) is.

The probability of the first song not being a repeat is quite good, so
the probability of this event happening is 1. The probability of the
seoncd song not being a repeat is also quite good, but one of the songs
is forbidden. The probability of having picked a song not yet played
before is therefore \(999/1000\). The third song is where we picked a
song we have already heard before. The probability of that happening is
quite slim - \(2/1000\). Altogether,
\(P(X = 3) = \frac{1000}{1000}\frac{999}{1000}\frac{2}{1000}\)

This is all well and good, but suppose we want to generalize this
equation for \(x\). Say we listened to \(x\) songs, where the \(x\)th
song was the duplicate. To generalize this, we can write

\(P(X=x) = \frac{1000}{1000}\frac{999}{1000}...\frac{1000-(x+2)}{1000}\frac{x-1}{1000}\)

The denominator can be simplified to \(1000^x\), and the top reeks of a
factorial\ldots to me it looks very much like the front end of a
factorial. And the front end of a factorial is simply a factorial with
the back end divided out. So the numerator is something like

\(1000*999*998*...*1000-(x+2) = 1000!/(1000-(x+1))!\)

So altogether it becomes

\(P(X = x) = \frac{1000!(x-1)}{1000^x(1000-(x+1))!}\)

\hypertarget{b-7}{%
\subsection{b}\label{b-7}}

\begin{center}\rule{0.5\linewidth}{0.5pt}\end{center}

It is much more computationally feasible the expected value if we define
the probability in a recursive form.

We can think of the question we want to answer as "what is the
probability of getting a repeat on this particular song number as well
as not getting a repeat on the previous songs? Mathematically:

\(P(X = k \cap X > k-1)\)

We know that

\(P(A|B) = \frac{P(A\cap B)}{P(B)}\)

by definition, so

\(P(A\cap B) = P(A|B)P(B)\)

so

\(P(X = k \cap X > k-1) = P(X = k|X > k-1)P(X > k-1)\)

We know that

\(P(X = k|X > k-1)\)

is just

\(\frac{k-1}{1000}\)

and

\(P(X > k-1)\)

is

\(1-(P(X=k-1) + P(X = k-2) + ... + P(X = 2))\)

So in total,

\(\frac{k-1}{1000}(1-(P(X=k-1) + P(X = k-2) + ... + P(X = 2)))\)

Thus we can define each probability incrementally instead of requiring
calculating \(1000!\), which can in turn be implemented into a loop.
These values can then be solved computationally to receive approximate
values rather than exact:

\begin{Shaded}
\begin{Highlighting}[]
\CommentTok{# Code shamefully stolen from the lab markdown}
\NormalTok{nsongs <-}\StringTok{ }\DecValTok{1000}

\NormalTok{xk <-}\StringTok{ }\DecValTok{1}\OperatorTok{:}\NormalTok{(nsongs }\OperatorTok{+}\StringTok{ }\DecValTok{1}\NormalTok{)}

\NormalTok{pk <-}\StringTok{ }\KeywordTok{rep}\NormalTok{(}\DecValTok{0}\NormalTok{, nsongs }\OperatorTok{+}\StringTok{ }\DecValTok{1}\NormalTok{)}

\ControlFlowTok{for}\NormalTok{ (k }\ControlFlowTok{in} \DecValTok{1}\OperatorTok{:}\KeywordTok{length}\NormalTok{(xk)) \{}
\NormalTok{    pk[k] <-}\StringTok{ }\NormalTok{(k }\OperatorTok{-}\StringTok{ }\DecValTok{1}\NormalTok{)}\OperatorTok{/}\NormalTok{nsongs }\OperatorTok{*}\StringTok{ }\NormalTok{(}\DecValTok{1} \OperatorTok{-}\StringTok{ }\KeywordTok{sum}\NormalTok{(pk[}\DecValTok{1}\OperatorTok{:}\NormalTok{(k }\OperatorTok{-}\StringTok{ }\DecValTok{1}\NormalTok{)]))}
\NormalTok{\}}

\NormalTok{EX <-}\StringTok{ }\KeywordTok{sum}\NormalTok{(pk }\OperatorTok{*}\StringTok{ }\NormalTok{xk)}
\NormalTok{VX <-}\StringTok{ }\KeywordTok{sum}\NormalTok{(pk }\OperatorTok{*}\StringTok{ }\NormalTok{xk}\OperatorTok{^}\DecValTok{2}\NormalTok{) }\OperatorTok{-}\StringTok{ }\NormalTok{EX}\OperatorTok{^}\DecValTok{2}
\end{Highlighting}
\end{Shaded}

\hypertarget{problem-11}{%
\section{Problem 11}\label{problem-11}}

\begin{center}\rule{0.5\linewidth}{0.5pt}\end{center}

This is essentially asking for a quantile of a geometric distribution.
\texttt{qgeom} will give us the number of `failures' (not contracting
HIV) before a `success' (contracting HIV) (Obviously these are just how
they are coded - no one would consider contracting HIV to be an actual
success).

\begin{Shaded}
\begin{Highlighting}[]
\KeywordTok{qgeom}\NormalTok{(}\FloatTok{0.1}\NormalTok{, }\DecValTok{1}\OperatorTok{/}\DecValTok{500}\NormalTok{)}
\end{Highlighting}
\end{Shaded}

\begin{verbatim}
## [1] 52
\end{verbatim}

We will have 52 `failures' before we have a 0.1 cumulative probability
of `success' - that is, 53 acts of intercourse are required. Some
assumptions of the geometric distribution: the events are independent -
that is, one event's probabilities does not affect the probability of
the other - and that there is no gradient of success and failure ie it
is a success/failure binary.

\hypertarget{problem-16}{%
\section{Problem 16}\label{problem-16}}

\begin{center}\rule{0.5\linewidth}{0.5pt}\end{center}

\hypertarget{a-8}{%
\subsection{a}\label{a-8}}

\begin{center}\rule{0.5\linewidth}{0.5pt}\end{center}

For \(M = 0, P(X = 0) = (408)/7745 \approx 0.053\) For
\(M = 1, P(X = 1) = (429 + 451 + 456 + 441)/7745 \approx 0.229\) For
\(M = 2, P(X = 2) = (497+486+473+526+498+490)/7745 \approx 0.384\) For
\(M = 3, P(X = 3) = (549+514+523+467)/7745 \approx 0.265\) For
\(M = 4, P(X = 4) = (537)/7745 \approx 0.069\)

\hypertarget{b-8}{%
\subsection{b}\label{b-8}}

\begin{center}\rule{0.5\linewidth}{0.5pt}\end{center}

This is the sum of each probability multiplied by X, so

\(E[X] = 0*0.053 + 1*0.229 + 2*0.384 + 3*0.265 + 4*0.069 = 2.068\)

\hypertarget{c-2}{%
\subsection{c}\label{c-2}}

\begin{center}\rule{0.5\linewidth}{0.5pt}\end{center}

The variance is given by

\(Var(X) = E[X^2] - E[X]^2\)

\(E[X^2] = 0^2*0.053 + 1^2*0.229 + 2^2*0.384 + 3^2*0.265 + 4^2*0.069 = 5.254\)
\(E[X]^2 = 2.068^2 = 4.277\)

\(Var(X) = 5.254 - 4.277 = 0.977\)

\hypertarget{d}{%
\subsection{d}\label{d}}

\begin{center}\rule{0.5\linewidth}{0.5pt}\end{center}

The conversion from number to proportion is as simple as converting
\(X = 0, 1, 2, 3, 4\) to \(X = 0, 0.25, 0.5, 0.75, 1\)

So

\(E[X] = 0*0.053 + 0.25*0.229 + 0.5*0.384 + 0.75*0.265 + 1*0.069 = 0.517\)

Which we note is simply the previous mean divided by 4

and

\(Var(X) = E[X^2] - E[X]^2\)

\(E[X^2] = 0^2*0.053 + 0.25^2*0.229 + 0.5^2*0.384 + 0.75^2*0.265 + 1^2*0.069 = .328375\)
\(E[X]^2 = 0.517^2 = 0.267289\)

\(Var(X) = .328375 - 0.267289 = 0.061086\)

Which we note is simply the previous variance divided by 16

\hypertarget{problem-17}{%
\section{Problem 17}\label{problem-17}}

\begin{center}\rule{0.5\linewidth}{0.5pt}\end{center}

\hypertarget{a-9}{%
\subsection{a}\label{a-9}}

\begin{center}\rule{0.5\linewidth}{0.5pt}\end{center}

\(E[X] = \int \limits_1^\infty x\frac{3}{x^4} = 3\int \limits_1^\infty \frac{1}{x^3} = -\frac{3}{2}x^{-2}\Big|^\infty_1 = \frac{3}{2}\)

The average failure time is 1.5 years

\hypertarget{b-9}{%
\subsection{b}\label{b-9}}

\begin{center}\rule{0.5\linewidth}{0.5pt}\end{center}

\(Var(X) = E[X^2] - E[X]^2\)

\(E[X^2] = \int \limits_1^\infty x^2\frac{3}{x^4} = 3\int \limits_1^\infty \frac{1}{x^2} = -3x^{-1}\Big|^\infty_1 = 3\)

\(Var(X) = 3 - 1.5^2 = 0.75\)

The variance is 0.75 years\^{}2

\hypertarget{c-3}{%
\subsection{c}\label{c-3}}

\begin{longtable}[]{@{}l@{}}
\toprule
\endhead
\begin{minipage}[t]{0.48\columnwidth}\raggedright
The general form of the Pareto distribution is given by\strut
\end{minipage}\tabularnewline
\begin{minipage}[t]{0.48\columnwidth}\raggedright
\(\frac{\beta\alpha^\beta}{x^{\beta+1}}\)\strut
\end{minipage}\tabularnewline
\begin{minipage}[t]{0.48\columnwidth}\raggedright
\(E[X] = \int \limits_\alpha^\infty x\frac{\beta\alpha^\beta}{x^{\beta+1}} = \beta\alpha^\beta\int \limits_\alpha^\infty \frac{1}{x^{\beta}} = -\frac{\beta\alpha^\beta x^{1-\beta}}{1-\beta}\Big|^\infty_\alpha = \frac{\beta\alpha}{\beta-1}\)\strut
\end{minipage}\tabularnewline
\begin{minipage}[t]{0.48\columnwidth}\raggedright
\(E[X^2] = \int \limits_\alpha^\infty x^2\frac{\beta\alpha^\beta}{x^{\beta+1}} = \beta\alpha^\beta\int \limits_\alpha^\infty \frac{1}{x^{\beta-1}} = \frac{\beta\alpha^\beta x^{-\beta+2}}{-\beta+2}\Big|^\infty_\alpha = \frac{\beta\alpha^2}{\beta-2}\)\strut
\end{minipage}\tabularnewline
\begin{minipage}[t]{0.48\columnwidth}\raggedright
\(Var(X) = E[X^2] - E[X]^2 = \frac{\beta\alpha^2}{\beta-2} - \Big(\frac{\beta\alpha}{\beta-1}\Big)^2\)\strut
\end{minipage}\tabularnewline
\begin{minipage}[t]{0.48\columnwidth}\raggedright
\# Problem 18\strut
\end{minipage}\tabularnewline
\bottomrule
\end{longtable}

\hypertarget{a-10}{%
\subsection{a}\label{a-10}}

\begin{center}\rule{0.5\linewidth}{0.5pt}\end{center}

The expected value can be calculated as

\(E[X] = Win*Prob + Loss*(1-Prob)\)

Where in this instance we assume a fair coin (\(Prob = 0.5\)) that has
flips that are independent (the outcome of one does not effect the
outcome of the other) and there can only be two outcomes - heads or
tails.

In this example, \(Win = \$1\) and \(Loss = -\$1\), such that

\(E[X] = \$1* 0.5 -\$1*0.5 = 0\)

On a long enough timeframe, regardless of the number of tosses, we
expect the true mean to be 0 - we will neither win nor lose.

\hypertarget{b-10}{%
\subsection{b}\label{b-10}}

\begin{center}\rule{0.5\linewidth}{0.5pt}\end{center}

\(Var(X) = E[X^2] - E[X]^2\)

\(E[X^2] = \$1^2*0.5+(-\$1)^2*0.5 = \$^21\)

\(E[X]^2 = 0\)

\(Var(X) = \$^21\)

\hypertarget{c-4}{%
\subsection{c}\label{c-4}}

\begin{center}\rule{0.5\linewidth}{0.5pt}\end{center}

In the case that \(p = 0.4\),

\(E[X] = \$1* 0.4 -\$1*0.6 = -\$0.2\)

\(E[X^2] = \$1^2*0.4+(-\$1)^2*0.6 = \$^21\)

\(E[X]^2 = \$^20.04\)

\(Var(X) = \$^20.96\)

\hypertarget{problem-19}{%
\section{Problem 19}\label{problem-19}}

\begin{center}\rule{0.5\linewidth}{0.5pt}\end{center}

\hypertarget{a-11}{%
\subsection{a}\label{a-11}}

\begin{center}\rule{0.5\linewidth}{0.5pt}\end{center}

\begin{Shaded}
\begin{Highlighting}[]
\NormalTok{temp <-}\StringTok{ }\KeywordTok{matrix}\NormalTok{(}\KeywordTok{sample}\NormalTok{(}\DecValTok{1}\OperatorTok{:}\DecValTok{6}\NormalTok{, }\DecValTok{1000} \OperatorTok{*}\StringTok{ }\DecValTok{10}\NormalTok{, }\DataTypeTok{replace =}\NormalTok{ T), }\DecValTok{1000}\NormalTok{)}
\NormalTok{xBar <-}\StringTok{ }\KeywordTok{apply}\NormalTok{(temp, }\DecValTok{1}\NormalTok{, mean)}
\end{Highlighting}
\end{Shaded}

\begin{Shaded}
\begin{Highlighting}[]
\KeywordTok{hist}\NormalTok{(xBar, }\DataTypeTok{breaks =} \DecValTok{100}\NormalTok{)}
\end{Highlighting}
\end{Shaded}

\includegraphics{/Users/kaiaragaki/Documents/github-repos/Biostats_140.651.01/output/hw_2_files/figure-latex/unnamed-chunk-14-1}

\hypertarget{b-11}{%
\subsection{b}\label{b-11}}

\begin{center}\rule{0.5\linewidth}{0.5pt}\end{center}

\begin{Shaded}
\begin{Highlighting}[]
\KeywordTok{mean}\NormalTok{(xBar)}
\end{Highlighting}
\end{Shaded}

\begin{verbatim}
## [1] 3.4929
\end{verbatim}

The mean of \texttt{xBar} is close to the expectation value for a die.
We would expect this to be true since the expectation value for a die is
indeed 3.5. When we we take samples with replacement from any
distribution and take the means, the means tend to be normally
distributed, and the mean of means tends to be centered on the mean of
the original population. This is due to the central limit.

\begin{Shaded}
\begin{Highlighting}[]
\KeywordTok{sd}\NormalTok{(xBar)}
\end{Highlighting}
\end{Shaded}

\begin{verbatim}
## [1] 0.5398621
\end{verbatim}

\hypertarget{c-5}{%
\subsection{c}\label{c-5}}

\begin{center}\rule{0.5\linewidth}{0.5pt}\end{center}

While the standard deviation might not look familiar, the variance
could:

\begin{Shaded}
\begin{Highlighting}[]
\KeywordTok{sd}\NormalTok{(xBar)}\OperatorTok{^}\DecValTok{2}
\end{Highlighting}
\end{Shaded}

\begin{verbatim}
## [1] 0.291451
\end{verbatim}

It looks to be about 1/10th of the true variance of a die
(\textasciitilde2.9167) - so we can expect the standard deviation to be
similar to sqrt(var/10):

\begin{Shaded}
\begin{Highlighting}[]
\KeywordTok{sqrt}\NormalTok{((}\KeywordTok{sum}\NormalTok{(((}\DecValTok{1}\OperatorTok{:}\DecValTok{6}\NormalTok{)}\OperatorTok{^}\DecValTok{2}\NormalTok{)}\OperatorTok{/}\DecValTok{6}\NormalTok{) }\OperatorTok{-}\StringTok{ }\FloatTok{3.5}\OperatorTok{^}\DecValTok{2}\NormalTok{)}\OperatorTok{/}\DecValTok{10}\NormalTok{)}
\end{Highlighting}
\end{Shaded}

\begin{verbatim}
## [1] 0.5400617
\end{verbatim}

\begin{Shaded}
\begin{Highlighting}[]
\KeywordTok{sd}\NormalTok{(xBar)}
\end{Highlighting}
\end{Shaded}

\begin{verbatim}
## [1] 0.5398621
\end{verbatim}

\hypertarget{problem-20}{%
\section{Problem 20}\label{problem-20}}

\begin{center}\rule{0.5\linewidth}{0.5pt}\end{center}

\hypertarget{a-12}{%
\subsection{a}\label{a-12}}

\begin{center}\rule{0.5\linewidth}{0.5pt}\end{center}

\begin{Shaded}
\begin{Highlighting}[]
\NormalTok{xBar <-}\StringTok{ }\KeywordTok{apply}\NormalTok{(}\KeywordTok{matrix}\NormalTok{(}\KeywordTok{runif}\NormalTok{(}\DecValTok{1000} \OperatorTok{*}\StringTok{ }\DecValTok{10}\NormalTok{), }\DecValTok{1000}\NormalTok{), }\DecValTok{1}\NormalTok{, mean)}
\end{Highlighting}
\end{Shaded}

\begin{Shaded}
\begin{Highlighting}[]
\KeywordTok{hist}\NormalTok{(xBar, }\DataTypeTok{breaks =} \DecValTok{100}\NormalTok{)}
\end{Highlighting}
\end{Shaded}

\includegraphics{/Users/kaiaragaki/Documents/github-repos/Biostats_140.651.01/output/hw_2_files/figure-latex/unnamed-chunk-20-1}
\#\# b

\begin{center}\rule{0.5\linewidth}{0.5pt}\end{center}

\begin{Shaded}
\begin{Highlighting}[]
\KeywordTok{mean}\NormalTok{(xBar)}
\end{Highlighting}
\end{Shaded}

\begin{verbatim}
## [1] 0.5008595
\end{verbatim}

As before, the mean of these means approaches the mean of the
distribution which it was sampled (the uniform distribution has a mean
of 0.5). Again this is because of the central limit theorem.

\hypertarget{c-6}{%
\subsection{c}\label{c-6}}

\begin{center}\rule{0.5\linewidth}{0.5pt}\end{center}

\begin{Shaded}
\begin{Highlighting}[]
\KeywordTok{sd}\NormalTok{(xBar)}
\end{Highlighting}
\end{Shaded}

\begin{verbatim}
## [1] 0.09002335
\end{verbatim}

Similarly, the reltionship is easier to make if we take the square and
compare it to the variance of the uniform distribution:

\begin{Shaded}
\begin{Highlighting}[]
\KeywordTok{sd}\NormalTok{(xBar)}\OperatorTok{^}\DecValTok{2}
\end{Highlighting}
\end{Shaded}

\begin{verbatim}
## [1] 0.008104204
\end{verbatim}

\begin{Shaded}
\begin{Highlighting}[]
\KeywordTok{sd}\NormalTok{(}\KeywordTok{runif}\NormalTok{(}\DecValTok{1000} \OperatorTok{*}\StringTok{ }\DecValTok{10}\NormalTok{))}\OperatorTok{^}\DecValTok{2}
\end{Highlighting}
\end{Shaded}

\begin{verbatim}
## [1] 0.08457111
\end{verbatim}

Again we note that the variance of this sampling distribution is close
to \(Var(X)/n\) where \(Var(X)\) is the variance of the distribution
from which samples are taken, and \(n\) is the number of `groups' that
are being aggregated into means.



\end{document}
